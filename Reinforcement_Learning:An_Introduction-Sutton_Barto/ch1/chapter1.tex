\documentclass{article}
\usepackage[margin=0.5in]{geometry}
\begin{document}
\textit{Excercise 1.1} Self-Play. Both sides will learn how to not lose the game since the counter part does not have flaws in the game strategy (learned to not lose the game either)

\textit{Excersize 1.2} Symmetries. We can group symmetric positions and choose between move (consider symmetries) and a random walk. Considering symmetries will decrease lerning time, since
will require less moves to learn optimal (in certain sense) strategy. We, probably, should not. Because the opponent could have flaws in certain symmetries, so the values will be higher.

\textit{Excersize 1.3} Greedy player can stuck in local minima and could never find the global one. Greedy player could be worse, but unlikely would be better, although it's possible that the greedy player
will find a global minima faster than non-greedy player.

\textit{Excersize 1.4} When we do not learn from the from exploratory moves the final set of probabilities will be sub-optimal and could represent local minima, learning from the exploratory moves
enables us to explore beyond the greedy solution.

\textit{Excersize 1.5} Since estimated probabilities converging to true proabbilities over the time, I think it's not possible to imrove the soltion to the tic-tac-toe problem as posed.

\end{document}
